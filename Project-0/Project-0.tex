\documentclass[onecolumn,draftclsnofoot, 10pt, compsoc]{IEEEtran}

\usepackage{graphicx}
\usepackage[section]{placeins}
\usepackage{caption}

\usepackage{amssymb}                                         
\usepackage{amsmath}                                         
\usepackage{amsthm}                                

\usepackage{alltt}                                           
\usepackage{float}
\usepackage{color}
\usepackage{url}

\usepackage{balance}
\usepackage[TABBOTCAP, tight]{subfigure}
\usepackage{enumitem}
\usepackage{pstricks, pst-node}
\usepackage{url}
\usepackage{setspace}

\usepackage{etoolbox}
\AtBeginEnvironment{quote}{\singlespacing\vspace{-\topsep}\small}

%\input{pygments.tex}

\usepackage{geometry}
\geometry{left=0.75in,right=0.75in,top=0.75in,bottom=0.75in}
\parindent = 0.0 in
\parskip = 0.1 in


\def \ParSpace{\vspace{.75em}}
\def \GroupNumber{		17}
\def \Jeremy{			Jeremy Fischer}
\def \Class{		Parallel Programming}
\def \Assn{		Project 0: Simple OpenMP Experiment}
\def \School{	Oregon State University}
\def \Professor{		Mike Bailey}

\newcommand{\cred}[1]{{\color{red}#1}}
\newcommand{\cblue}[1]{{\color{blue}#1}}

\newcommand{\NameSigPair}[1]{
		\par
		\makebox[2.75in][r]{#1} \hfil 	\makebox[3.25in]{\makebox[2.25in]{\hrulefill} \hfill			
		\makebox[.75in]{\hrulefill}}
		\par\vspace{-12pt} \textit{
			\tiny\noindent
			\makebox[2.75in]{} \hfil		
			\makebox[3.25in]{
				\makebox[2.25in][r]{Signature} \hfill	\makebox[.75in][r]{Date}
			}
		}
}










%%%%%%%%%%%%%%%%%%%%%%%%%%%%%%%%%%%%%%%
\begin{document}
\begin{titlepage}
    \pagenumbering{gobble}
    \begin{singlespace}
    	\includegraphics[height=4cm]{coe.eps}
        \hfill  
        \par\vspace{.2in}
        \centering
        \scshape{
            \vspace{.5in}
            \textbf{\Huge\Assn}\par
            \textbf{\Large\Class}\par
            \large{
            	\today \\Spring Term
        	}
            \vfill
            {\large Prepared for}\par
            \huge \School\par
            \vspace{5pt}
            {\Large{\Professor}\par}
            {\large Prepared by }\par
           % Group\GroupNumber\par
            \vspace{5pt}
            {\Large
                {\Jeremy}\par
            }
            \vspace{20pt}
        }

    \end{singlespace}
\end{titlepage}
\newpage
\pagenumbering{arabic}

% 7. uncomment this (if applicable). Consider adding a page break.
%\listoffigures
%\listoftables
\clearpage


	\section{Tell what machine you ran this on}	
	I ran this test on a 2015 Macbook Pro, 2.2 GHz Intel Core i7, 16 GB 1600 MHz DDR3.
	
	
	
	\section{ What performance results did you get?}
	With \textit{ARRAYSIZE = 10000} and \textit{NUMTRIES = 30} I got \dots
	
	\underline{1 Thread}
	\\
	Peak Performance =  2000.26 MegaMults/Sec
	\\
	Average Performance =  1924.50 MegaMults/Sec
	
	\underline{4 Threads}
	\\
	Peak Performance =  243.90 MegaMults/Sec
	\\
	Average Performance =  181.65 MegaMults/Sec

	
	\section{ What was your 4-thread-to-one-thread speedup?}
	My 4-thread-to-one-thread speedup was 10.60.
	
	
	
	
	\section{ Why do you think it is behaving this way?}
	Multiplying numbers from two arrays and storing them in a result array is a situation where threads can be used to their full advantage, because there is no overlap between multiplications. Meaning, each thread can take a portion of the multiplications and compute them. This increases the performance of the computation.
	
	
	
	
	\section{ What was your Parallel Fraction, Fp?}
	My parallel fraction, Fp = 1.207.
	
	
	
	



\end{document}